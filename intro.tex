This paper can be edited by multiple authors (similar to a google doc). It can also be cloned as a git repo and edited on your computer. You can then push your code to github and view your changes on Authorea. PI's need not fear the document can also be downloaded as a Word doc or PDF for review (although as more academics read content in web browsers it may make sense to learn to write your work in a web friendly manner using tools like Authorea).


Intro from Authorea: 
\begin{quote}
At Authorea, we want to change the way scientists communicate and share their research. This includes giving all the information behind figures a place to live: by letting readers and reviewers access your data and code, your results can be easily reproduced and extended.
\end{quote}

\begin{quote}
It's really easy to incorporate IPython Notebooks in your articles.  First, upload a figure to your article. You can do so by dragging and dropping an image or by clicking \textbf{Insert Figure} at the bottom of the block you want it under.  
\end{quote}

\begin{quote}
When your figure is in place you can \textbf{attach data} to it. We take json, csv, dat files and \textbf{IPython Notebooks}. 
\end{quote}