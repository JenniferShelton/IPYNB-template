If you add a dataset in json, csv, or dat format, the data will be included in the figure folder. Find it by going to the folder view, and then browsing to the figures folder.

If you attach an IPython Notebook, you also get the notebook to be included in the figure folder. And in addition to that, your readers can also: 
\begin{enumerate}
\item \textbf{Launch IPython} directly in their browsers (by clicking on the link below the figure);
\item see your annotated code and data;
\item test it;
\item adjust it to their pleasing;
\item download it.
\end{enumerate}
Beyond the obvious \textit{advantages }this provides for streamlining the scientific process, imagine implementing this to facilitate classroom learning or centralizing repeated analysis in a \textit{lab setting.}  What's more, it gives you a place to share and be very descriptive with your code.  

In the predator-prey modeling example below (no data, just a model), a detailed walk-through is given in the IPython Notebook. The hope is that anyone so inclined could modify or fork it, perhaps adding a third organism or other environmental constraints.  Or, if they had relevant ecological data, to test to see how well the model fits. Go check it out!
